\chapter{Quantum Electrodynamics (QED)}
\setcounter{chapter}{5}
\section{Classical Electrodynamics and Maxwell's equations}
We have the gauge potential $A^\mu = (A^0, \pmb{A}) = (\phi, \pmb{A})$ \& $A_\mu = (A^0, -\pmb{A}) = (\phi, -\pmb{A})$ and the field strength tensor $F_{\mu\nu} = \partial_\mu A_\nu - \partial_\nu A_\mu$.

Then
\begin{itemize}
	\item electric field $E_i = F_{0i} = \partial_0 A_i - \partial_i A_0 \rightarrow \pmb{E} = -\dot{\pmb{A}} - \pmb{\nabla} \phi$
	\item magnetic field $B^i = -\frac{1}{2} \epsilon^{ijk}F_{jk} \rightarrow \pmb{B} = \pmb{\nabla} \times \pmb{A}$
\end{itemize}

Lagrangian density $\lag_{EM} = -\frac{1}{4}F_{\mu\nu}F^{\mu\nu} = -\frac{1}{2} (\pmb{E}\cdot\pmb{E} - \pmb{B}\cdot \pmb{B})$. The field equation $\partial_\mu \left( \frac{\partial \lag}{\partial(\partial_\mu A_\nu)} \right) - \frac{\partial \lag}{\partial A_\nu} = 0$ leads to
\begin{align}
	\partial_\mu F^{\mu\nu} = 0
\end{align}
it is half of Maxwell's equations (in vacuum).

The other half are Bianchi identities following from the definition of $F_{\mu\nu}$:
\begin{align*}
	\partial_\lambda F_{\mu\nu} + \partial_\mu F_{\nu\lambda} + \partial_\nu F_{\lambda\mu} = 0 \Leftrightarrow \epsilon^{\sigma \lambda \mu \nu}\partial_\lambda F_{\mu\nu} = 0\\
	\text{or } \partial_\lambda \tilde{F}^{\sigma\lambda} = 0,\; \tilde{F}^{\sigma\lambda} = \frac{1}{2} \epsilon^{\sigma\lambda\mu\nu}F_{\mu\nu}
\end{align*}

In terms of $\pmb{E}$ and $\pmb{B}$:
\begin{align*}
	\pmb{\nabla}\cdot \pmb{E} = 0&,\; \dot{\pmb{E}} = \pmb{\nabla}\times \pmb{B} \quad \text{dynamical equations} \\
	\pmb{\nabla}\cdot \pmb{B} = 0&,\; \dot{\pmb{B}} = -\pmb{\nabla}\times \pmb{E} \quad \text{Bianchi identities}
\end{align*}

Remarks
\begin{itemize}
	\item Lagrangian density does not depend on $\dot{A}_0$, since $A_0$ is not really dynamical.
	\begin{align*}
		\pmb{\nabla}\cdot \pmb{E} = 0 \rightarrow \pmb{\nabla}^2 A_0 + \pmb{\nabla}\cdot \dot{\pmb{A}} = 0 
	\end{align*}
	Solve this \underline{Poisson} equation for $A_0(\pmb{x},t) = \frac{1}{4\pi}\int \dd^3 y \frac{\pmb{\nabla}\cdot \dot{\pmb{A}}(\pmb{y},t)}{|\pmb{y}-\pmb{x}|}$. Thus $A_0$ is given in terms of the other components of $A$.
	\item \underline{gauge invariance}: field strength tensor invariant under the transformation $A_\mu \longmapsto A_\mu - \partial_\mu \text{X}$ due to commuting derivatives. This leads to gauge invariance of Maxwell equations.\\
	Choose X to satisfy $\partial_\mu \partial^\mu \text{X} = \partial^2\text{X}=\partial_\mu A^\mu$ allows us to demand the condition  (Lorenz condition)
		\begin{align}
			\partial_\mu A^\mu = 0
		\end{align}
	such that $A_\mu$ belongs to the "Lorenz gauge" and reduces the degrees of freedom from 4 to 3.
	\begin{itemize}
		\item Further freedom is eliminated by adding any X with $\partial^2 \text{X} = 0$, e.g.~$\frac{\partial}{\partial t} \text{X} = A_0$. Then we get the \underline{Coulomb} or \underline{radiation gauge}
		\begin{align}
			A_0 = 0,\; \pmb{\nabla}\cdot \pmb{A} = 0
		\end{align}
	\end{itemize}
\end{itemize}

Note: vice versa imposing $\pmb{\nabla} \cdot \pmb{A} = 0$ first, yields $A_0 = 0$ (using Lorenz condition?).

In Coulomb gauge:
\begin{align*}
	&\pmb{E} = -\dot{\pmb{A}}.\; \pmb{B} = \pmb{\nabla} \times \pmb{A},\; \pmb{\nabla}\times \pmb{A} = 0 \\
	&-\ddot{\pmb{A}} = \dot{\pmb{E}} \stackrel{\text{Maxwell}}{=} \pmb{\nabla} \times \pmb{B} =  \pmb{\nabla} \times (\pmb{\nabla} \times \pmb{A}) = \pmb{\nabla}(\underbrace{\pmb{\nabla}\cdot\pmb{A}}_{=0}) - \pmb{\nabla}^2 \pmb{A} \\
	&\Rightarrow \partial^2 \pmb{A} = 0
\end{align*}
This wave equation is massless KG equation for each spatial component.

Then the solutions are obvious: $\pmb{A} = \pmb{\epsilon} e^{-ik\cdot x}$ with $k^2=0$ and $\pmb{\epsilon}\cdot\pmb{k}=0$. The polarization vector $\pmb{\epsilon}$ is transverse to $\pmb{k}$.

Can write the lagrangian in Coulmb gauge 
\begin{align*}
	\lag_\text{EM} = \frac{1}{2}\dot{\pmb{A}}\dot{\pmb{A}} - \frac{1}{2} \pmb{B}\cdot \pmb{B}
\end{align*}
Then the conjugate momentum to $\pmb{A}$ is $\pmb{\Pi} = \frac{\partial \lag}{\partial \dot{A}} = \dot{\pmb{A}} = -\pmb{E}$. It has only 3 components, there is no conjugate momentum to $A_0$!. Because of Coulomb gauge $\pmb{\Pi}$ is subject to the constraint $\pmb{\nabla}\cdot\pmb{\Pi} = 0$

Hamiltonian
\begin{align*}
	H_{\text{EM}} = \int \dd^3 x \left( \frac{1}{2} \pmb{\Pi}\cdot\pmb{\Pi} + \frac{1}{2} \pmb{B}\cdot\pmb{B} \right)
\end{align*}

\section{Quantizing the Maxwell field}
We would like to impose canonical commutation relations, à la
\begin{align*}
	&[ A_i(\pmb{x}), A_j (\pmb{y}) ] =  [ \Pi_i(\pmb{x}), \Pi_j (\pmb{y}) ] = 0 \\
	&[A_i(\pmb{x}), \Pi_j(\pmb{y})] = i \delta_{ij} \delta^{(3)} (\pmb{x} - \pmb{y})
\end{align*}

However this cannot be true. Take either derivative of the last equation and it needs to vanish due to $\pmb{\nabla}\cdot\pmb{A} = \pmb{\nabla}\cdot \pmb{\Pi} = 0$. But 
\begin{align*}
	[\partial^i A_i (\vecx), \Pi_k(\vecy)] =  i \delta_{ij} \partial^i \delta^{(3)}(\vecx-\vecy)
\end{align*}
here the derivative is takev with respect to $\vecx$, i.e.~$\partial^i = \frac{\partial}{\partial x_i}$.


Replace $\delta_{ij}$ by $\Delta_{ij}$
\begin{align*}
	& [\partial^i A_i (\vecx), {\Pi}_{j}(\vecy)] = i \Delta_{ij} \partial^i \frac{1}{(2\pi)^3} \int \dd^3 k e^{i\pmb{k}\cdot(\vecx - \vecy)} \\
	& = -\frac{1}{(2\pi)^3} \int \dd^3 k (k^i\Delta_{ij}) e^{i\pmb{k}\cdot(\pmb{x}-\pmb{y})} \stackrel{!}{=} 0
\end{align*}
it works for $\Delta_{ij} = \delta_{ij}-\frac{k_ik_j}{\pmb{k}^2}$ in momentum space or $\Delta_{ij} = \delta_{ij} - \vecnab^{-2}\partial_i \partial_j$ in position space.

\begin{align}
	[A_i(\pmb{x}), \Pi_j(\pmb{y})] = i \left( \delta_{ij} - \vecnab^{-2}\partial_i \partial_j \right) \delta^{(3)} (\pmb{x} - \pmb{y})
\end{align}

As before we have the mode expansion
\begin{align*}
	\pmb{A}(\vecx) &= \int \frac{\dd^3 k}{(2\pi)^3\sqrt{2|\veck|}} \left( \pmb{a}_{\veck}e^{i\veck\cdot\vecx} + \pmb{a}^\dagger_{\veck} e^{-i\veck\cdot\vecx} \right) \\
	\pmb{\Pi}(\vecx) &= \int \frac{\dd^3 k}{(2\pi)^3} (-i)\sqrt{\frac{|\veck|}{2}}\left( \pmb{a}_{\veck}e^{i\veck\cdot\vecx} - \pmb{a}^\dagger_{\veck} e^{-i\veck\cdot\vecx} \right) \\
\end{align*}
with $\veck\cdot\pmb{a}_{\veck} = \veck\cdot\pmb{a}^\dagger_{\veck} = 0$.

Introduce 2 orthogonal polarization vectors $\pmb{\epsilon}^{(1)}(\veck)$ and $\pmb{\epsilon}^{(2)}(\veck)$ for each $\veck$.
\begin{align*}
	&\pmb{a}_{\veck} = a_{\veck}^{(1)}\pmb{\epsilon}^{(1)} + a_{\veck}^{(2)}\pmb{\epsilon}^{(2)} = \sum_{\lambda=1}^2 a_{\veck}^{(\lambda)}\pmb{\epsilon}^{(\lambda)}(\veck) \\
	& \text{with } \veck\cdot\pmb{\epsilon}^{(1)}(\veck) = \veck\cdot\pmb{\epsilon}^{(2)}(\veck) = 0,\; \pmb{\epsilon}^{(\lambda)}\cdot \pmb{\epsilon}^{(\lambda;)}=\delta_{\lambda \lambda'}
\end{align*}

Creation and annihilation operator have the standard commutation relations
\begin{align}
	[a_{\veck}^{(\lambda)}, a_{\veck'}^{(\lambda')\dagger}] = (2\pi)^3 \delta_{\lambda\lambda'}\delta^{(3)}(\veck-\veck')
\end{align} 
and all other commutators vanish. Geometrically, still possible to write including the unphysical longitudinal components:
\begin{align*}
	[\pmb{a}_{\veck}, \pmb{a}_{\pmb{l}}] &= 	[\pmb{a}_{\veck}^\dagger, \pmb{a}_{\pmb{l}}^\dagger] = 0 \\
	[a^i_{\veck}, a_{\pmb{l}}^{j\dagger}] &= (2\pi)^3 \left( \delta^{ij} - \frac{k^i k^j}{\veck^2} \right) \delta^{(3)}(\veck-\pmb{l})
\end{align*}

$a_{\veck}^{(\lambda)}$ and $a_{\veck}^{(\lambda)\dagger}$ create and destroy photons of momentum $\veck$, energy $|\veck|$ and (electric) polarization along $\pmb{\epsilon}^{(\lambda)}(\veck)$. 

Next steps are analogout to KG theory. 
\paragraph{Hamiltonian}
\begin{align*}
	H &= \frac{1}{2} \int \dd^3 x \left(\pmb{E}^2 + \pmb{B}^2 \right)= \frac{1}{2} \int \dd^3 x \left( \dot{\pmb{A}}^2 + (\vecnab \times \pmb{A})\cdot (\vecnab \times \pmb{A}) \right) \\
	\shortintertext{using identity $\pmb{A}\cdot(\pmb{B}\times\pmb{C}) = \pmb{B}\cdot(\pmb{C}\times\pmb{A})$}
	  &= \frac{1}{2} \int \dd^3 x \left( \dot{\pmb{A}}^2 + \pmb{A}\cdot\vecnab\times(\vecnab \times \pmb{A})\right) \\
	  \shortintertext{using the identity $\vecnab\times(\vecnab\times\pmb{A}) = \vecnab(\vecnab\cdot\pmb{A} - \vecnab^2\pmb{A})$}
	  &= \frac{1}{2} \int \dd^3 x \left( \dot{\pmb{A}}^2 -\pmb{A}\cdot\vecnab^2\pmb{A}+\pmb{A}\cdot\vecnab(\vecnab\cdot\pmb{A})  \right) \\
	\shortintertext{using coulomb gauge condition}
	  &= \frac{1}{2} \int \dd^3 x \left( \dot{\pmb{A}}^2 - \pmb{A}\cdot \nabla^2\pmb{A} \right) \\
	  \shortintertext{the first term vanishes and use normal ordering}
	  &= \int \frac{\dd^3 k}{(2\pi)^3} \left|\veck \right| \pmb{a}_{\veck}^\dagger \cdot \pmb{a}_{\veck} = \sum_{\lambda=1}^2 \int \frac{\dd^3 k}{(2\pi)^3} \left|\veck \right| a_{\veck}^{(\lambda \dagger)} a_{\veck}^{\lambda}
\end{align*} 

\paragraph{Heisenberg field}
\begin{align*}
	\pmb{A}(\pmb{x},t) = \int \frac{\dd^3 k}{(2\pi)^3}\frac{1}{\sqrt{2|\pmb{k}|}} \left( \pmb{a}_{\pmb{k}} e^{-ik\cdot x} +  \pmb{a}_{\pmb{k}}^\dagger e^{ik\cdot x}\right)
\end{align*}

\paragraph{Photon propagator}
\begin{align}
	\braket{0 | T A_i (x) A_j (y) | 0} \eqdef D^{\text{tr}}_{ij} (x-y) = \int\frac{\dd^4 k}{(2\pi)^4}	
\frac{i}{k^2 + i\epsilon} \left( \delta_{ij} - \frac{k_i k_j}{|\pmb{k}|^2} \right)e^{-ik\cdot(x-y)}
\end{align}
$\text{tr}$ stands for transverse: photon polarization perpendicular to its momentum. This is \textcolor{red}{NOT} the final version of the photon propagator!

\section{Inclusion of matter - QED}
\begin{align}
	\lag_\text{QED} &= -\frac{1}{4}F_{\mu\nu}F^{\mu\nu} + \bar{\psi} (i\slashed{D} - m)\psi
	\shortintertext{where $D_\mu = \partial_\mu + ieA_\mu$ is the (gauge) covariante derivative}
					&= \lag_\text{EM} + \lag_{D} -e \underbrace{\bar\psi \gamma^\mu \psi A_\mu}_{j^\mu}
\end{align}

Field equations would be
\begin{align*}
	\partial_\mu F^{\mu\nu} = e j^\nu \qquad (i\slashed{D}-m) \psi = 0
\end{align*}
where $ej^\nu$ is the electromagnetic 4-current.

Gauge invariance under the transformation
\begin{align*}
	\begin{cases}
		\psi(x) \longmapsto \psi'(x) = e^{ie\chi(x)}\psi \\
		A_\mu(x) \longmapsto A'_\mu(x) = A_\mu(x) - \partial_\mu \chi(x)
	\end{cases}
\end{align*}

To check the consistence: cavariant derivative transforms like $D_\mu \longmapsto D'_\mu\psi'(x) = e^{ie\chi(x)}D_\mu \psi(x)$. Since the adjoint spinor transforms like $\bar{\psi}(x) \longmapsto \bar{\psi}'(x) = \bar{\psi}(x) e^{-ie\chi(x)}$, the Lagrangian and field equations are gauge invariant.

Again we choose Coulomb gauge $\vecnab \cdot \pmb{A} = 0$, then equation for $A^0$:
\begin{align}
	\partial_i F^{i0} &= ej^0 \notag\\ 
	\Rightarrow -\vecnab^2 A^0 &= ej^0 =  e \bar{\psi}\gamma^0 \psi \notag\\
							   &= e \bar{\psi}\gamma^0\psi = e\psi^\dagger\psi \notag\\
							   &= e \rho(x)\notag\\
	A^0(\pmb{x},t) &= e \int \dd^3 y \frac{\rho(\pmb{y}, t)}{4\pi \left| \pmb{x} - \pmb{y} \right|  }
\end{align}

%%%%%%%%%%%%%%%%%%%%%%%%%%%%%%%%%%%%%%%%%%%%%%

We want to derive the interaction Hamiltonian. Note 
\begin{align*}
	\int \dd^3 x \frac{1}{2} \pmb{E}^2 &= \int \dd^3 x \frac{1}{2} \left(\dot{\pmb{A}} + \vecnab A^0 \right)^2 \\
	\shortintertext{cross terms vanish after integration by parts due to $\vecnab \cdot \dot{\pmb{A}} = 0$}
									   &= \int \dd^3 x \frac{1}{2} \left(\dot{\pmb{A}}^2 + (\vecnab A^0)^2 \right) \\
									   &= \int \dd^3 x \frac{1}{2} \left(\dot{\pmb{A}}^0 - A^0 \vecnab^2 A^0 \right) \\
									   \shortintertext{$-e j^0 = -e \rho$}
									   &= \int \dd^3 x \frac{1}{2} \dot{\pmb{A}}^2 + \underbrace{\frac{e^2}{2} \int \dd^3 x \dd^3 y \frac{\rho(\pmb{x})\rho(\pmb{y})}{4\pi |\pmb{x}-\pmb{y}|}}_{=\frac{e^2}{2} j^0 A_0}
\end{align*}

Combined Hamiltonian
\begin{align*}
	H =& \int \dd^3 x \left\{ \frac{1}{2}\pmb\Pi\cdot \pmb\Pi + \frac{1}{2} \pmb{B}\cdot \pmb{B} + i\bar\psi \pmb{\gamma}\cdot \vecnab \psi + m \bar\psi \psi \right\}  \quad \text{free photon and fermion} \\
	   &+ \frac{e^2}{2} \int \dd^3 x \dd^3 y \frac{\rho(\pmb{x})\rho(\pmb{y})}{4\pi |\pmb{x} - \pmb{y}|} - e \int \dd^3 x \pmb{j} \cdot \pmb{A} \quad \text{interactions}
\end{align*}
where $\rho = \psi^\dagger \psi = \bar\psi \gamma^0 \psi$, $\pmb{j} = \bar\psi \pmb{\gamma}\psi$ for 2 types of interactions.

\section{Lorentz-invariant propagator}
Consider $e^- e^-$ scattering at $\mathcal{O}(e^2)$
\begin{align*}
	\feynmandiagram[small, horizontal=x to y, baseline=(x.base)]{p3[label=\(p_3\)] --[anti fermion] x[dot] --[photon, reversed momentum=\(k\)] y[dot] --[fermion] p4[label=\(p_4\)], p1[label=\(p_1\)] --[fermion] x, y--[anti fermion]p2[label=\(p_2\)]}; \quad k = p_3 - p_1 = p_2 - p_4
\end{align*}

We expect this to involve 
\begin{itemize}
	\item spinors for external fermions
	\item $-ie\gamma^\mu$
	\item Photon propagator $D_{\mu\nu}(x-y)$
\end{itemize}

What we have found in Coulomb gauge is actually
\begin{itemize}
	\item %TODO diagram
		vertices $ie\gamma^i$, transverse propagator $D^\text{tr}_{\mu\nu}(x-y)$
	\item %TODO diagram
		vertices $\pm ie \gamma^0$, instantaneous Coumlomb interaction $\frac{1}{4\pi |\pmb{x}- \pmb{y}|}\delta(x^0 - y^0)$
\end{itemize}

Effectively combine these propagators terms into $D_{\mu\nu}^\text{Coul}(x-y)$, where the  $D_{00}^\text{Coul}(x-y) = \frac{1}{4\pi |\pmb{x}- \pmb{y}|}\delta(x^0 - y^0)$.  This component in momentum space is simply 
$$\int \frac{\dd^3 k}{(2\pi)^3} \frac{e^{i\pmb{k}\cdot \pmb{r}}}{|\pmb{k}|^2} = \frac{1}{4\pi|\pmb{r}|}$$

Therefore Coumlomb propagator in momentum space:
\begin{align*}
	D^\text{Coul}_{\mu\nu}(k) = 
	\begin{cases}
		\frac{i}{k^2 + i \epsilon} \left(\delta_{ij} - \frac{k_ik_j}{|\pmb{k}|^2} \right) & \mu = i, \nu=j \\
		\frac{i}{|\pmb{k}|^2} & \mu = \nu = 0\\
		0 & \text{otherwise}
	\end{cases}
\end{align*}

Consider contraction to scattering amplitude from vertex at x:
\begin{align*}
	%TODO diagram
	\sim e \bar{u} (p_3) \gamma^\mu u(p_1) e^{i(p_3 - p_1)x} \\
	\shortintertext{current conservation $\partial_\mu j^\mu = 0$ written in momentum space}
	\underbrace{(p_3 - p_1)_\mu}_{k_\mu} \bar{u} (p_3) \gamma^\mu u(p_1) = 0
\end{align*}
so in the complete diagram $D^\text{Coul}_{\mu\nu}$ occurs in a form %TODO: How exactly? and why a and b are transverse to k?
\begin{align*}
	& a^\mu D^\text{Coul}_{\mu\nu}(k) b^\nu \\
	=& a^0 \frac{i}{|\pmb{k}|^2}b^0 +  a^i \left[ \frac{i}{k^2 + i \epsilon} \left(\delta_{ij} - \frac{k_ik_j}{|\pmb{k}|^2} \right) \right]  b^j \\
	\shortintertext{where $k^\mu a^\mu = 0, k_\mu a^\mu =0$}
	=& i \Bigg[ \frac{\pmb{a}\cdot\pmb{b}}{k^2} {\underbrace{- \frac{k^2_0 a_0 b_0}{k^2 |\pmb{k}|^2} + \frac{a_0 b_0}{|\pmb{k}|^2} }_{=\frac{-k_0^2 a_0 b_0 + a_0 b_0 (k_0^2 - |\pmb{k}|^2)}{k^2|\pmb{k}|}^2}} \Bigg] \\
	=& \frac{i}{k^2}(\pmb{a}\cdot\pmb{b}-a_0b_0) = -\frac{i}{k} a_\mu b^\mu
\end{align*}

\paragraph{Conclusion}
in this diagram (and in fact, in general), we may replace the $D^\text{Coul}_{\mu\nu}(k)$ by the manifestly Lorentz covariant propagator 
\begin{align}
	D_{\mu\nu} (k) = - \frac{i g_{\mu\nu}}{k+i\epsilon}
\end{align}

This can be generalised to 
\begin{align}
	D_{\mu\nu}(k) = -\frac{i}{k^2 + i \epsilon} \left( g_{\mu\nu} - (1-\alpha)\frac{k_\mu k_\nu}{k^2} \right) 
	\shortintertext{as, by current consevation, additional term doesn't contribute.} \notag
\end{align}
Feynman gauge $\alpha = 1$; Landau gauge $\alpha=0$.

\paragraph{Remark} one can also try to quantise photons in a manifestly covariant way, imposing Lorentz gauge $\partial_\mu A^\mu = 0$
\begin{align*}
	[A_\mu (\pmb{x}), \Pi_\nu (\pmb{y})] = ig_{\mu\nu} \delta^{(3)} (\pmb{x} - \pmb{y})
\end{align*}
This is trouble since $\Pi^0 = \frac{\partial \lag}{\partial \dot{A}_0} = 0$. This cannot hold!

We thus change the Lagrangian $$\lag = - \frac{1}{4} F_{\mu\nu} F^{\mu\nu} - \frac{1}{2\alpha} (\partial_\mu A^\mu)^2$$ with "gauge fixing term". The equation of motion from this is $$\partial^2 A^\mu - \left(1-\frac{1}{\alpha}\right)\partial^\mu (\partial_\lambda A^\lambda) = 0$$ e.g.~ $\alpha=1$ is the Feynman gauge.

With this Lagrangian we can the 0th component of conjugate momentum $$\Pi^0 = -\frac{1}{\alpha} \partial_\mu A^\mu$$ but this seems as bad as before!

We cannot impose Coulomb gauge condition $\partial_\mu A^\mu = 0$ as an operator identity. Instead demand a weaker condition $\braket{\text{out}|\partial_\mu A^\mu|\text{in}} = 0$ for all physical states.

This in turn tells us which states are actually physical. The 4 polarisation states consist of physical, timelike(scalar) and longitudinal states. The negative-norm states cancel each ther out (Gopta-Bleuler formalism).

\paragraph{Feynman rules for QED}
diagrams constucted from electron (positron) \feynmandiagram[horizontal=a to b,small]{a--[fermion]b}; and photon \feynmandiagram[horizontal=a to b,small]{a--[photon]b};; rules for fermions are valid as before.

In addition
\begin{itemize}
	\item vertex $\feynmandiagram[small, horizontal=x to y, baseline=(x.base)]{a--[anti fermion]x[dot, label=45:\(\mu\)]--[photon]y, b--[fermion]x}; = -ie\gamma^\mu$;
	\item photon propagator $\feynmandiagram[horizontal=a to b,small]{a[label=\(\mu\)]--[photon, edge label=\(k\)]b[label=\(\nu\)],}; =-\frac{ig_{\mu\nu}}{k^2 + i \epsilon}$
	\item external photons
		\begin{align*}
			\feynmandiagram[horizontal=a to b,small]{a[label=\(\mu\), dot]--[reversed momentum={\(k_\text{in}\)}, photon]b[label=\(\nu\)],}; = \epsilon_\mu \\
			\feynmandiagram[horizontal=a to b,small]{a[label=\(\mu\), dot]--[momentum={\(k_\text{out}\)}, photon]b[label=\(\nu\)],}; = \epsilon_\nu^*
		\end{align*}
		$\epsilon_\mu$ polarisation vector of in/out photon and $\epsilon^*_\mu$ for out photon required for complex (circular) polarisation.
\end{itemize}

\section{QED process at tree level}
\paragraph{Example $e^-e^- \rightarrow e^- e^-$}
\begin{align*}
	&\feynmandiagram[small, horizontal=x to y, baseline=(x.base)]{
		p3[label=90:\(p_3\)] --[anti fermion] x --[photon] y --[fermion] p4[label=90:\(p_4\)],
		p1[label=-90:\(p_1\)] --[fermion] x,
		y --[anti fermion] p2[label=-90:\(p_2\)],
	};
	&&\begin{tikzpicture}[baseline=(x.base)]
	\begin{feynman}
		\diagram[small, horizontal=x to y]{
		p3[label=90:\(p_3\)] --[draw=none] x --[photon] y --[draw=none] p4[label=90:\(p_4\)],
		p1[label=-90:\(p_1\)] --[fermion] x,
		y --[anti fermion] p2[label=-90:\(p_2\)],
		};
		\diagram*{
			(p3) -- [anti fermion] (y),
			(x) --[fermion] (p4),
		};
	\end{feynman}
	\end{tikzpicture}\\
	&\sim \frac{1}{(p_1 - p_3)^2} = \frac{1}{t}  
	&&\sim  \frac{1}{(p_1 - p_4)^2} = \frac{1}{u}  \\
	&\quad \text{t-channel} &&\quad \text{u-channel}
\end{align*}

\paragraph{Example $e^-e^+ \rightarrow e^- e^+$}
\begin{align*}
	&\feynmandiagram[small, horizontal=x to y, baseline=(x.base)]{
		p3[label=90:\(p_3\)] --[anti fermion] x --[photon] y --[fermion] p4[label=90:\(p_4\)],
		p1[label=-90:\(p_1\)] --[fermion] x,
		y --[anti fermion] p2[label=-90:\(p_2\)],
	};
	&& \feynmandiagram[small, vertical=x to y, baseline=(x.base)]{
		p4[label=90:\(p_4\)] --[anti fermion] x --[photon] y --[fermion] p2[label=-90:\(p_2\)],
		p3[label=90:\(p_3\)] --[fermion] x,
		y --[anti fermion] p1[label=-90:\(p_1\)],
	}; \\
	&\sim \frac{1}{(p_1 - p_3)^2} = \frac{1}{t}
	&&\sim \frac{1}{(p_1 + p_2)^2} = \frac{1}{s} \\
	&\text{t-channel} && \text{s-channel}
\end{align*}

\paragraph{Compton scattering $\gamma e^- \rightarrow \gamma e^-$}
\begin{align*}
	&\feynmandiagram[small, vertical=x to y, baseline=(x.base)]{
		p4[label=90:\(p_4\)] --[anti fermion] x --[anti fermion] y --[anti fermion] p2[label=-90:\(p_2\)],
		p3[label=90:\(p_3\)] --[photon] x,
		y --[photon] p1[label=-90:\(p_1\)],
	};
	&&\begin{tikzpicture}[baseline=(x.base)]
	\begin{feynman}
		\diagram[small, vertical'=x to y]{
		p3[label=90:\(p_3\)] --[draw=none] x --[anti fermion] y --[draw=none] p1[label=-90:\(p_1\)],
		p4[label=90:\(p_4\)] --[anti fermion] x,
		y --[anti fermion] p2[label=-90:\(p_2\)],
		};
		\diagram*{
			(p3) -- [photon] (y),
			(x) --[photon] (p1),
		};
	\end{feynman}
	\end{tikzpicture} \\
	&\text{s-channel} &&\text{u-channel}
\end{align*}

\paragraph{Example $e^+ e^- \rightarrow \gamma \gamma$}
\begin{align*}
	&\feynmandiagram[small, horizontal=x to y, baseline=(x.base)]{
		p3 --[photon] x --[fermion] y --[photon] p4,
		p1[label=-90:\(e^-\)] --[fermion] x,
		y --[fermion] p2[label=-90:\(e^+\)],
	};
	&&\begin{tikzpicture}[baseline=(x.base)]
	\begin{feynman}
		\diagram[small, horizontal=x to y]{
		p3 --[draw=none] x --[fermion] y --[draw=none] p4,
		p1[label=-90:\(e^-\)] --[fermion] x,
		y --[fermion] p2[label=-90:\(e^+\)],
		};
		\diagram*{
			(p3) -- [photon] (y),
			(x) --[photon] (p4),
		};
	\end{feynman}
	\end{tikzpicture}\\
	&\quad \text{t-channel} &&\quad \text{u-channel}
\end{align*}
These are important for lifetime of positronium.

All these amplitudes are $\mathcal{O}(e^2)$, $\alpha = \frac{e^2}{4\pi}=\frac{1}{137.036}$ the fine structure constant.

Muons $\mu^\pm$, like electrons, just ca. 200 times heavier.
\begin{align*}
e^- \mu^- \rightarrow e^- \mu^- \Rightarrow
\feynmandiagram[small, horizontal=x to y, baseline=(x.base)]{
		p3[label=90:\(e^-\)] --[anti fermion] x --[photon] y --[fermion] p4[label=90:\(\mu^-\)],
		p1[label=-90:\(e^-\)] --[fermion] x,
		y --[anti fermion] p2[label=-90:\(\mu^-\)],
}; \quad \text{Just one diagram}
\end{align*}

\begin{align*}
e^+ e^- \rightarrow \mu^+ \mu^- \Rightarrow
\feynmandiagram[small, vertical=x to y, baseline=(x.base)]{
		p3[label=90:\(\mu^+\)] --[fermion] x --[photon] y --[fermion] p4[label=-90:\(e^+\)],
		p1[label=90:\(\mu^-\)] --[anti fermion] x,
		y --[anti fermion] p2[label=-90:\(e^-\)],
};\quad \text{Just one diagram}
\end{align*}
$\mu^\pm$ decay into $e^\pm$ and neutrinos in weak interactions.

For tree level diagrams the photon propagator does not need to have the $i\epsilon$ in the denominator, since we will never be able to see a singularity/pole.

\subsection{Some hints and tricks for cross section calculations} See application in exercises!
\paragraph{Example}
$e^+ e^- \rightarrow \mu^+ \mu^-$ \\
\begin{minipage}{0.4\textwidth}
	\begin{center}
	\feynmandiagram[small, vertical'=x to y,baseline=(x.base)]{
		a[label=\(\mu^-\)] --[anti fermion, edge label=\(k\)] x[dot] --[photon] y[dot] --[anti fermion, edge label=\(p\)] c[label=-90:\(e^-\)],
		b[label=\(\mu^+\)] --[fermion, edge label=\(k'\)] x,
		y --[fermion, edge label=\(p'\)] d[label=-90:\(e^+\)],
	};
	\end{center}
\end{minipage}%
\begin{minipage}{0.6\textwidth}
	\begin{align*}
		iM &= \bar{\nu}_e^{s'}(-ie\gamma^\mu) u_e^s(p) \left. \frac{-ig_{\mu\nu}}{s}\right|_{s=q^2} \bar{u}^r_{\mu}(k) (-ie\gamma^\nu) \nu^{r'}_{\mu}(k') \\
		&= \frac{ie^2}{s} \left( \bar\nu_e(p')\gamma^\mu u_e(p) \right) \left( \bar{u}_\mu(k) \gamma_\mu \nu_\mu(k') \right)
	\end{align*}
\end{minipage}

See section \ref{sec:croSec}, $|M|^2$ is needed for cross section. $M^*$ involves things like 
\begin{align*}
	(\bar{\nu}\gamma^\mu u)^* &= (\bar{\nu} \gamma^\mu u)^\dagger = u^\dagger \gamma^{\mu\dagger}\gamma_0^\dagger \nu \\
							  &= u^\dagger \gamma_0 \gamma^\mu \gamma_0\gamma_0\nu = \bar{u} \gamma^\mu \nu
\end{align*}

So 
$$|M|^2 = \frac{e^4}{s^2} \left[ \bar{v}(p') \gamma^\mu u (p) \bar{u}(p)\gamma^\nu v(p') \right]_{e^\pm} \cdot  \left[ \bar{u}(k) \gamma_\mu v (p) \bar{v}(k')\gamma_\nu u(k) \right]_{\mu^\pm}$$
Unpolarized scattering$=\frac{1}{4} \sum_{r,s,r',s'}|M|^2$.

Now $\bar{v}\gamma^\mu u$, $\bar{u}\gamma^\nu v$ etc. are scalars in Dirac/spinor space:
\begin{align*}
	&\sum_{s,s'} \bar{v}_{s'}{p'} \gamma^\mu u_s(p) \bar{u}_s (p) \gamma^\nu v_{s'} (p') \\
	\text{(taking trace of scalar) }	=& \sum_{s,s'} \text{Tr} \left(\bar{v}_{s'}{p'} \gamma^\mu u_s(p) \bar{u}_s (p) \gamma^\nu v_{s'} (p') \right) \\
	=& \sum_{s,s'} \text{Tr} \left( v_{s'}(p')\bar{v}_{s'}(p') \gamma^\mu u_s(p) \bar{u}_s(p)\gamma^\nu \right) \\
	\shortintertext{using spin sums}
	=& \text{Tr} \left( (\slashed{p}' - m)\gamma^\mu (\slashed{p}+m)\gamma^\nu \right)
\end{align*}

\paragraph{Trace technology}
\begin{itemize}
	\item remember $\text{Tr} \gamma_\mu = 0 $
	\item $\text{Tr}(\gamma^\mu \gamma^\nu) = 4g_{\mu\nu}$
	\item $\text{Tr}(\text{odd number of } \gamma) = 0$
	\item $\text{Tr}(\gamma_\mu \gamma_\nu \gamma_\alpha \gamma_\beta) = 4 \left( g_{\mu\nu}g_{\alpha\beta} + g_{\mu\beta}g_{\nu\alpha} - g_{\mu\alpha} g_{\nu\beta} \right)$
	\item more rules involving $\gamma_5$ (weak interactions!)
\end{itemize}

So
\begin{align*}
	\text{Tr} \left( (\slashed{p}' - m)\gamma^\mu (\slashed{p}+m)\gamma^\nu \right) &= 4 \left( p'^\mu p^\nu + p'^\nu p^\mu - g^{\mu\nu}(p\cdot p' + m^2) \right)
\end{align*}

\paragraph{Mandelstam variables}
with 4 equal masses, center-of-mass system (CMS): 
$$p=(E,\pmb{p}), \quad p'=(E,-\pmb{p}),\quad k = (E, \pmb{k}),\quad \theta = \measuredangle (\pmb{p}, \pmb{k})$$

\begin{align}
	s &\stackrel{\text{CMS}}{=} (p+p')^2 = 4E^2 \\
	t &=  (p-k)^2 = -(\pmb{p}-\pmb{k})^2 = -2 |\pmb{p}|^2 (1-\cos{\theta}) \\
	u &= (p' - k)^2 = -2 |\pmb{p}|^2 (1+\cos{\theta})\\
	|\pmb{p}|^2 &= E^2 - m^2 = \frac{s}{4} - m^2
\end{align}

Only 2 Mandelstam variables are independent.
\begin{align*}
	s + t + u &= (p_1 + p_2)^2 + (p_1 - p_3)^2 + (p_1 - p_4)^2 \\
				&= p_1^2 + p_2^2 +p_3^2 + p_4^2 + 2 p_1 \underbrace{(p_1 + p_2 - p_3 - p_4)}_{=0} \\
				&= \sum_i m_i^2 = \text{const}
\end{align*}

\paragraph{photon polarisation sums}
Analogy to fermion spin sums before, Feynman rules for external photons involve $\epsilon^(*)_\mu$; e.g.~Compton amplitude of the form 
\begin{align*}
	M \sim \epsilon^*_\mu (p_3) \epsilon_\nu (p_1) T^{\mu\nu}
\end{align*}

Thus 
\begin{align*}
	\sum_{\text{spin, pol.}} |M|^2 =  \sum_{\text{spin, pol.}} \epsilon^*_\mu (p_3) \epsilon_\alpha (p_3) \epsilon^*_\beta (p_1) \epsilon_\nu (p_1) T^{\mu\nu} T^{\alpha\beta^*}
\end{align*}
How can we simplify $\sum_{pol} \epsilon_\mu^*(k) \epsilon_\nu(k)$? Again we have only 2 physical polarisation states, but want to do it in a covariant form.

Assume a simpler process (than Compton) with a single external photon, $\epsilon^*_\mu (k) M^\mu$. Choose $$k^\mu = (k,0,0,k),\quad \epsilon_{(1)}^\mu = (0,1,0,0),\quad \epsilon_{(2)}^\mu = (0,0, 1,0)$$

so $\sum_\text{pol}|\epsilon_\mu^*(k) M^\mu|^2 = |M_1|^2 + |M_2|^2$

Remember that photon coupled source $j^\mu$, current sonvervation $\partial_\mu j^\mu = 0$. We will see (next term) this holds in general as \underline{Ward identity}
\begin{align}
	k_\mu M^\mu = 0
\end{align}

In exercises, show $p_{3 \mu} T^{\mu\nu} = 0 = p_{1\nu}T^{\mu\nu}$ for Compton

Here $kM^0 - k M^3 = 0 \Rightarrow M^0 = M^3$ and we can rewrite 
\begin{align*}
	\sum_{pol} \epsilon_\mu^* \epsilon_\nu M^\mu M^{*\nu} = |M_1|^2 + |M_2|^2 + \underbrace{|M_3|^2 - |M_0|^2}_{=0} = -g_{\mu\nu} M^\mu M^{*\nu}
\end{align*}

so effectively
\begin{align}
	\sum_{pol} \epsilon^*_\mu (k) \epsilon_\nu (k) = -g_{\mu\nu}
\end{align}

\paragraph{side remark}
\begin{itemize}
	\item KG propagator $\frac{i}{p^2 - M^2 + i\epsilon}$
	\item Dirac propagator $\frac{i(\slashed{p}+m)}{p^2 - m^2 + i\epsilon} = \frac{i \sum_s u_s(p) \bar{u}_s(p)}{p^2 - m^2 + i\epsilon}$
	\item Photon propagator $\frac{-ig_{\mu\nu}}{p^2 + i\epsilon} = \frac{i\sum_{pol}\epsilon^*_\mu (p)\epsilon_\nu(p)}{p^2+i\epsilon}$
\end{itemize}
